\subsection{Endpoints}

%cabanias

\textbf{Endponint}:/cabanias

metodo: GET

recibe: res = cabania.obtener-cabanias()


\textbf{Endpoint}: /cabanias/calendario/<string:cabania-id>

metodo: GET

recibe: res = cabania.calendario-reservas(cabania-id)


\textbf{Endpoint}: /cabanias

metodo: POST 

recibe:
("secreto","id","nombre","descripcion","cap-max","precio-noche")


\textbf{Endpoint}: /cabanias/<string:id>

metodo: PATCH 

recibe: ("secreto","nombre","descripcion","cap-max","precio-noche")

metodo: DELETE

recibe: id (por URI)


%reservas


\textbf{Endpoint}: /reservas

metodo: GET

recibe:

("codigo-reserva": codigo-reserva,"nombre-cliente": nombre-cliente,

"cliente-id": cliente-id,"email": email,

"telefono": telefono, "fecha-ingreso": fecha-ingreso,

"fecha-egreso": fecha-egreso,"precio-total": precio-total)


\textbf{Endpoint}: /reserva

metodo: GET

recibe: ("cliente-id": cliente-id|pasaporte,"nombre-cliente" : nombre-cliente)


\textbf{Endpoint}: /crear-reserva

metodo: POST

recibe: ("cabania-id" : id,"fecha-ingreso" : "aaaa-mm-dd",

"fecha-egreso" : "aaaa-mm-dd","nombre-cliente" : "nombre-cliente",

"cliente-id": cliente-id,"telefono": telefono,

"email" : "example@example.com")


\textbf{Endpoint}: /reserva/<int:id>

metodo: DELETE

recibe:
("email" : example@mail.com)

Si "email" es proporcionado, la reserva del cliente con el codigo de reserva 'id', sera eliminada.


\textbf{Endpoint}: /reserva/<int:id>

metodo: PATCH

recibe: ("secreto" : passw)

luego se puede modificar lo que se quiera de la reserva

%imagenes

\textbf{Endpoint}: /imagenes

metodo: GET

recibe:("secreto" : passw, "cabania-id" : id *opcional*)

\textbf{Endpoint}: /crear-imagen

metodo: POST

recibe:

("secreto" : passw, "cabania-id" : id, *opcional*

"link" : url-img,"descripcion" : portada)

\textbf{Endpoint}: /imagen

metodo: DELETE

recibe:( "secreto" : passw, "link" : url)

\href{https://github.com/MaxiFttInst/UBA_TP_IDS/blob/api/api/app.py}{API del proyecto}


























